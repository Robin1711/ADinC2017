\documentclass[a4paper]{article}

\usepackage{a4wide,times}
\usepackage[english]{babel}

% -----------------------------------------------
% especially use this for you code
% -----------------------------------------------

\usepackage{courier}
\usepackage{listings}
\usepackage{color}

\definecolor{Gray}{gray}{0.95}

\lstset{ %
	language = C,                   % choose the language of the code
	basicstyle = \small\ttfamily,   % the size and fonts that are used
	frame = single,                 % adds a frame around the code
	tabsize = 3,                    % sets default tabsize
	breaklines = true,              % sets automatic line breaking
	numbers = left,                 % where to put the line-numbers
	numberstyle = \footnotesize,    % the style of the line-numbers
	backgroundcolor = \color{Gray}  % the background color of the listing
}

\lstdefinestyle{stdio}{
    basicstyle = \small\ttfamily,	% Font of the text
    frame = tb,				        % Style of the surrounding frame
    framextopmargin=.75mm,         % Space margin top
    framexbottommargin=.75mm,      % Space margin bottom
    framexleftmargin=2mm,          % Space margin left
    framexrightmargin=2mm,         % Space margin right
    tabsize = 3,					% Size of tab character
    breaklines = true,             % Wrap lines of text that are too long
    columns = flexible,			
    showstringspaces = false,
    backgroundcolor = \color{Gray}
}

% -- until here ---------------------------------

\begin{document}


\title{Programming report \\
       assignment 1 for A\&DinC
}
\date{\today}
\author{Thomas Janssen (s1565001) \\
        	Robin Sommer (s2997592)
}

\maketitle

\section{Problem description}

The problem was as follows. We were given two sets of sentences and we needed to compute for each sentence in the second set which sentences of the first set are an anagram of it. An anagram is a sentence which exists of the same characters.

\section{Problem analysis}

An anagram is a sentence which contains the same amount of characters and the amount of each different  character is also the same. By checking if the sentences have the same amount of characters is the first step to see whether they could be anagrams of each other. If the amount of characters isn't the same it cannot be an anagram of the sentence. If the amount is the same we go to the next step. To determine whether they really are anagrams you should check if the characters in the sentences are exactly the same.

\section{Program design}

First it is important to store the given sentences correctly. We chose to use 2D-arrays for this. First we scan the integer which stands for the amount of sentences that will be given. This enables us to create an array with that given size. Then we call a function which stores the given input (sentences) in the 2D array. We do this in such a way that the sentences are stored as histograms. So the amount of each character in the sentence is stored. The simplest way to do this is to read the character and subtract the character 'a' of it. In this way we generate an array where the a is on the first index and the z is on the last. We use the last index to store the amount of characters in that particular sentence. The second array for the test sentences will be generated in the same way.\\
The next step is to compare the test sentence with every sentence in the first set.
We use a double for-loop for this, where the outer loop represents the test sentences and the inner loop the sentences from the first set. Within the inner loop we check whether the amount of characters in the sentences are the same. If it is, we check whether the characters are also the same. We do this by comparing the two histograms. If they are equal we print the number of the sentence of the first set.\\
And last, but not least, we free the two array's that we have used.

\newpage

\section{Evaluation of the program}

We used the original test set which was provided with the assignment. We slightly altered it for testing the program extensions.

\begin{itemize}

\item Input:
\begin{lstlisting}[style = stdio]
7 
astronomers.
leaving your idol.
moon starers.
the fine game of nil.
avoiding our yell.
no.
lin grad.
4
no more stars.
none.
the meaning of life.
darling i love you.
\end{lstlisting}

 Output: (for anagram.c)
\begin{lstlisting}[style = stdio]
1 3 

4 
2 5  
\end{lstlisting}

 Output: (for anagram$\_$ext1.c)
\begin{lstlisting}[style = stdio]
no more stars. -  astronomers. 
no more stars. - moon starers. 
the meaning of life. - the fine game of nil. 
darling i love you. - leaving your idol. 
darling i love you. - avoiding our yell. 
\end{lstlisting}

 Output: (for anagram2$\_$extension2.c)
\begin{lstlisting}[style = stdio]
Test sentence 1, word 1 - sentence 6
Test sentence 1, words 1 till 3 - sentence 1
Test sentence 1, words 1 till 3 - sentence 3
Test sentence 3, words 1 till 4 - sentence 4
Test sentence 4, word 1 - sentence 7
Test sentence 4, words 1 till 4 - sentence 2
Test sentence 4, words 1 till 4 - sentence 5
\end{lstlisting}

\end{itemize}

We also checked the programs with valgrind and there were no memory leaks.

\section{Extension of the program (optional)}

We did both of the suggested extension for the program. For the first extension we needed to use an additional array to store the sentences. This way we could print them to the screen in the output. The comparison of the sentences remained the same. Only this time we print the test sentence and sentence with which it is an anagram.\\

\newpage

For the second extension we had to modify the array for the histograms of the test sentences. We now used a 3d-array. So we now could store histograms for each word in the test sentence. We use slightly adjusted functions for this. We also had to use an additional function to read the input for the test sentences. In that function we now first store the sentence in an array and then use that array to create the histograms for every word in the sentence. Next, before we compare the histograms we now add the histograms. So that each histogram of the words contains all the histograms of previous words in the test sentence. 
Now we can compare an initial segment of the test sentence with all the sentences from the first set.
The result of this is printed to the screen.\\
Finally we had to add an extra function to free the memory of the 3d-array.

\section{Process description}

This programming assignment went well. Except for one segmentation fault we haven't really encountered any problems. The most difficult part was probably not creating any segmentation fault by allocating and freeing the memory correctly.

\section{Conclusions}

Our program solves the problem correctly. We had a minor problem with a segmentation fault, but we fixed that. The first and the second extensions also work correctly.\\
The program itself is very efficient, because it directly generates a histogram of the sentences. There is no time and space wasted to make an extra array to first store the sentence and after that make an histogram.
In extension 1 we do make use of a second array to store the sentence. But it is still efficient because while scanning the sentence we store the sentence and make a histogram at the same time.\\
In extension 2 we had to store the sentences first and then make the histograms. But only for the test sentences.
\\
We think this is the optimal solution to the problem.\\

\section{Appendix: program text}

\lstinputlisting[caption = \tt anagram.c]{anagram.c}

\newpage

\section{Appendix: extended program text (optional)}

\lstinputlisting[caption = \tt anagram$\_$ext1.c]{anagram_ext1.c}

\newpage

\lstinputlisting[caption = \tt anagram2extension2.c]{anagram2extension2.c}

\end{document}